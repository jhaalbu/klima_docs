%% Generated by Sphinx.
\def\sphinxdocclass{report}
\documentclass[letterpaper,10pt,norsk]{sphinxmanual}
\ifdefined\pdfpxdimen
   \let\sphinxpxdimen\pdfpxdimen\else\newdimen\sphinxpxdimen
\fi \sphinxpxdimen=.75bp\relax
\ifdefined\pdfimageresolution
    \pdfimageresolution= \numexpr \dimexpr1in\relax/\sphinxpxdimen\relax
\fi
%% let collapsible pdf bookmarks panel have high depth per default
\PassOptionsToPackage{bookmarksdepth=5}{hyperref}

\PassOptionsToPackage{booktabs}{sphinx}
\PassOptionsToPackage{colorrows}{sphinx}

\PassOptionsToPackage{warn}{textcomp}
\usepackage[utf8]{inputenc}
\ifdefined\DeclareUnicodeCharacter
% support both utf8 and utf8x syntaxes
  \ifdefined\DeclareUnicodeCharacterAsOptional
    \def\sphinxDUC#1{\DeclareUnicodeCharacter{"#1}}
  \else
    \let\sphinxDUC\DeclareUnicodeCharacter
  \fi
  \sphinxDUC{00A0}{\nobreakspace}
  \sphinxDUC{2500}{\sphinxunichar{2500}}
  \sphinxDUC{2502}{\sphinxunichar{2502}}
  \sphinxDUC{2514}{\sphinxunichar{2514}}
  \sphinxDUC{251C}{\sphinxunichar{251C}}
  \sphinxDUC{2572}{\textbackslash}
\fi
\usepackage{cmap}
\usepackage[T1]{fontenc}
\usepackage{amsmath,amssymb,amstext}
\usepackage{babel}



\usepackage{tgtermes}
\usepackage{tgheros}
\renewcommand{\ttdefault}{txtt}



\usepackage[Sonny]{fncychap}
\ChNameVar{\Large\normalfont\sffamily}
\ChTitleVar{\Large\normalfont\sffamily}
\usepackage{sphinx}

\fvset{fontsize=auto}
\usepackage{geometry}


% Include hyperref last.
\usepackage{hyperref}
% Fix anchor placement for figures with captions.
\usepackage{hypcap}% it must be loaded after hyperref.
% Set up styles of URL: it should be placed after hyperref.
\urlstyle{same}

\addto\captionsnorsk{\renewcommand{\contentsname}{Contents:}}

\usepackage{sphinxmessages}
\setcounter{tocdepth}{1}



\title{AV\sphinxhyphen{}Klima}
\date{jan. 25, 2023}
\release{}
\author{Jan Helge Aalbu}
\newcommand{\sphinxlogo}{\vbox{}}
\renewcommand{\releasename}{}
\makeindex
\begin{document}

\ifdefined\shorthandoff
  \ifnum\catcode`\=\string=\active\shorthandoff{=}\fi
  \ifnum\catcode`\"=\active\shorthandoff{"}\fi
\fi

\pagestyle{empty}
\sphinxmaketitle
\pagestyle{plain}
\sphinxtableofcontents
\pagestyle{normal}
\phantomsection\label{\detokenize{index::doc}}



\chapter{Skildring av tjenesten}
\label{\detokenize{index:skildring-av-tjenesten}}
\sphinxAtStartPar
Tjenesten er ein web applikasjon for å forenkle uthenting av klimadata. Den nyttar
seg av eit gridda datasett, dvs eit datasett der det finnes data for kvar 1x1 km rute
av landet.

\sphinxAtStartPar
Fra NVE sin skredfareveileder (\sphinxurl{https://veileder-skredfareutredning-bratt-terreng.nve.no/})
er det åpna for at gridda data kan vere aktuelle å bruke, særleg der det er langt fra stajonsdata.
Ofte mangler nærliggande stasjonar også f.eks vind, snømengde eller andre aktuelle parametere.

\sphinxAtStartPar
Tjenesten gjer ein spørring til eit NVE API for å hente ut data for ein gitt koordinat. Den
lager så automatisk til plot med data som er aktuelle i ei skredfarevurdering. Eller andre
klimanalaysar.

\sphinxAtStartPar
Lenke til NVE api:
\sphinxurl{http://api.nve.no/doc/gridtimeseries-data-gts/}

\sphinxAtStartPar
Lenke til forklaring for generering av datasett:
\sphinxurl{https://senorge.no/Models}
Der er det vidare henvisning til vitenskaplige artiklar som skildrar metodane i detalj


\chapter{Bruksanvisning}
\label{\detokenize{index:bruksanvisning}}
\sphinxAtStartPar
Appen på web med lenke: \sphinxurl{https://app-avtools-klima-dev.azurewebsites.net/}

\sphinxAtStartPar
Den har enkelt input med trykk i kart, eller innskriving av koordinat.

\sphinxAtStartPar
Videre gir den moglegheit for å velje forskjellige type sampleplot

\sphinxAtStartPar
Klimaoversikt:
\sphinxhyphen{} Gjennomsnittleg månedstemperatur og nedbør
\sphinxhyphen{} Fordeling av snømengde gjennom året, med snitt, max og min sammen med temperatur
\sphinxhyphen{} Åresnedbør tilbake i tid
\sphinxhyphen{} Årleg maksimal snødjupne tilbake i tid

\sphinxAtStartPar
Klimaoversikt med 3 døgn snø og returverdi:
Dette gir dei samme plotta som for klimaoversikt me i tilleg
\sphinxhyphen{} 3 døgns snø mengde
\sphinxhyphen{} Returverdi for 3 døgns snø basert på gumbelfordeling

\sphinxAtStartPar
Dersom ein ynskjer å få verdier plotta ut på plot for gjennomsnittleg måndestemperatur og nedøbr er det mulig å huka av for å vise tall på normalplott

\sphinxAtStartPar
Dersom ein ynskjer ei vindanalyse for klimapunktet kan ein huke av for å køyre klimanalayse. Det blir da lasta ned datasett for vind og følgande plott blir generert:
\sphinxhyphen{} Vindretning for alle


\chapter{Nedhenting og bearbeiding av klimadata modul}
\label{\detokenize{index:module-klimadata.klimadata}}\label{\detokenize{index:nedhenting-og-bearbeiding-av-klimadata-modul}}\index{modul@\spxentry{modul}!klimadata.klimadata@\spxentry{klimadata.klimadata}}\index{klimadata.klimadata@\spxentry{klimadata.klimadata}!modul@\spxentry{modul}}\index{hent\_data\_klima\_dogn() (i modul klimadata.klimadata)@\spxentry{hent\_data\_klima\_dogn()}\spxextra{i modul klimadata.klimadata}}

\begin{fulllineitems}
\phantomsection\label{\detokenize{index:klimadata.klimadata.hent_data_klima_dogn}}
\pysigstartsignatures
\pysiglinewithargsret{\sphinxcode{\sphinxupquote{klimadata.klimadata.}}\sphinxbfcode{\sphinxupquote{hent\_data\_klima\_dogn}}}{\emph{\DUrole{n}{lat}\DUrole{p}{:}\DUrole{w}{  }\DUrole{n}{str}}, \emph{\DUrole{n}{lon}\DUrole{p}{:}\DUrole{w}{  }\DUrole{n}{str}}, \emph{\DUrole{n}{startdato}\DUrole{p}{:}\DUrole{w}{  }\DUrole{n}{str}}, \emph{\DUrole{n}{sluttdato}\DUrole{p}{:}\DUrole{w}{  }\DUrole{n}{str}}, \emph{\DUrole{n}{parametere}\DUrole{p}{:}\DUrole{w}{  }\DUrole{n}{list}}}{{ $\rightarrow$ dict}}
\pysigstopsignatures
\sphinxAtStartPar
Henter ned klimadata basert på liste av parametere
\begin{quote}\begin{description}
\sphinxlineitem{Parametere}\begin{itemize}
\item {} 
\sphinxAtStartPar
\sphinxstyleliteralstrong{\sphinxupquote{lat}} \textendash{} øst\sphinxhyphen{}vest koordinat (i UTM33)

\item {} 
\sphinxAtStartPar
\sphinxstyleliteralstrong{\sphinxupquote{lon}} \textendash{} nord\sphinxhyphen{}sør koordinat (i UTM33)

\item {} 
\sphinxAtStartPar
\sphinxstyleliteralstrong{\sphinxupquote{startdato}} \textendash{} startdato for dataserien som hentes ned

\item {} 
\sphinxAtStartPar
\sphinxstyleliteralstrong{\sphinxupquote{sluttdato}} \textendash{} sluttdato for dataserien som hentes ned

\item {} 
\sphinxAtStartPar
\sphinxstyleliteralstrong{\sphinxupquote{parametere}} \textendash{} liste med parametere som skal hentes ned f.eks rr for nedbør

\end{itemize}

\sphinxlineitem{Returnere}
\sphinxAtStartPar
dict med parameternavn til key, og liste med verdier som value

\sphinxlineitem{Retur type}
\sphinxAtStartPar
parameterdict

\end{description}\end{quote}

\end{fulllineitems}

\index{hent\_hogde() (i modul klimadata.klimadata)@\spxentry{hent\_hogde()}\spxextra{i modul klimadata.klimadata}}

\begin{fulllineitems}
\phantomsection\label{\detokenize{index:klimadata.klimadata.hent_hogde}}
\pysigstartsignatures
\pysiglinewithargsret{\sphinxcode{\sphinxupquote{klimadata.klimadata.}}\sphinxbfcode{\sphinxupquote{hent\_hogde}}}{\emph{\DUrole{n}{lon}\DUrole{p}{:}\DUrole{w}{  }\DUrole{n}{str}}, \emph{\DUrole{n}{lat}\DUrole{p}{:}\DUrole{w}{  }\DUrole{n}{str}}}{{ $\rightarrow$ str}}
\pysigstopsignatures
\sphinxAtStartPar
Henter ned høgdeverdi for koordinat fra NVE api
\begin{quote}\begin{description}
\sphinxlineitem{Parametere}\begin{itemize}
\item {} 
\sphinxAtStartPar
\sphinxstyleliteralstrong{\sphinxupquote{lat}} \textendash{} øst\sphinxhyphen{}vest koordinat (i UTM33)

\item {} 
\sphinxAtStartPar
\sphinxstyleliteralstrong{\sphinxupquote{lon}} \textendash{} nord\sphinxhyphen{}sør koordinat (i UTM33)

\end{itemize}

\sphinxlineitem{Returnere}
\sphinxAtStartPar
høgdeverdi for koordinat

\sphinxlineitem{Retur type}
\sphinxAtStartPar
høgde

\end{description}\end{quote}

\end{fulllineitems}

\index{klima\_dataframe() (i modul klimadata.klimadata)@\spxentry{klima\_dataframe()}\spxextra{i modul klimadata.klimadata}}

\begin{fulllineitems}
\phantomsection\label{\detokenize{index:klimadata.klimadata.klima_dataframe}}
\pysigstartsignatures
\pysiglinewithargsret{\sphinxcode{\sphinxupquote{klimadata.klimadata.}}\sphinxbfcode{\sphinxupquote{klima\_dataframe}}}{\emph{\DUrole{n}{lat}}, \emph{\DUrole{n}{lon}}, \emph{\DUrole{n}{startdato}}, \emph{\DUrole{n}{sluttdato}}, \emph{\DUrole{n}{parametere}}}{{ $\rightarrow$ DataFrame}}
\pysigstopsignatures\begin{description}
\sphinxlineitem{Lager dataframe basert på klimadata fra NVE api. Bruker underfunksjoner. Bruker}
\sphinxAtStartPar
start og sluttdato for å generere index i pandas dataframe.

\end{description}
\begin{quote}\begin{description}
\sphinxlineitem{Parametere}\begin{itemize}
\item {} 
\sphinxAtStartPar
\sphinxstyleliteralstrong{\sphinxupquote{lat}} \textendash{} øst\sphinxhyphen{}vest koordinat (i UTM33)

\item {} 
\sphinxAtStartPar
\sphinxstyleliteralstrong{\sphinxupquote{lon}} \textendash{} nord\sphinxhyphen{}sør koordinat (i UTM33)

\item {} 
\sphinxAtStartPar
\sphinxstyleliteralstrong{\sphinxupquote{startdato}} \textendash{} startdato for dataserien som hentes ned

\item {} 
\sphinxAtStartPar
\sphinxstyleliteralstrong{\sphinxupquote{sluttdato}} \textendash{} sluttdato for dataserien som hentes ned

\item {} 
\sphinxAtStartPar
\sphinxstyleliteralstrong{\sphinxupquote{parametere}} \textendash{} liste med parametere som skal hentes ned f.eks rr for nedbør

\end{itemize}

\sphinxlineitem{Returnere}
\sphinxAtStartPar
Pandas dataframe med klimadata

\sphinxlineitem{Retur type}
\sphinxAtStartPar
df

\end{description}\end{quote}

\end{fulllineitems}

\index{maxdf() (i modul klimadata.klimadata)@\spxentry{maxdf()}\spxextra{i modul klimadata.klimadata}}

\begin{fulllineitems}
\phantomsection\label{\detokenize{index:klimadata.klimadata.maxdf}}
\pysigstartsignatures
\pysiglinewithargsret{\sphinxcode{\sphinxupquote{klimadata.klimadata.}}\sphinxbfcode{\sphinxupquote{maxdf}}}{\emph{\DUrole{n}{df}\DUrole{p}{:}\DUrole{w}{  }\DUrole{n}{DataFrame}}}{{ $\rightarrow$ DataFrame}}
\pysigstopsignatures
\sphinxAtStartPar
Tar in klimadataframe, og returnerer ny dataframe med årlige maksimalverdier
\begin{quote}\begin{description}
\sphinxlineitem{Parametere}
\sphinxAtStartPar
\sphinxstyleliteralstrong{\sphinxupquote{df}} \textendash{} Pandas dataframe med klimadata

\sphinxlineitem{Returnere}
\sphinxAtStartPar
Pandas dataframe med årlige maksimalverdier

\sphinxlineitem{Retur type}
\sphinxAtStartPar
maxdf

\end{description}\end{quote}

\end{fulllineitems}

\index{nve\_api() (i modul klimadata.klimadata)@\spxentry{nve\_api()}\spxextra{i modul klimadata.klimadata}}

\begin{fulllineitems}
\phantomsection\label{\detokenize{index:klimadata.klimadata.nve_api}}
\pysigstartsignatures
\pysiglinewithargsret{\sphinxcode{\sphinxupquote{klimadata.klimadata.}}\sphinxbfcode{\sphinxupquote{nve\_api}}}{\emph{\DUrole{n}{lat}\DUrole{p}{:}\DUrole{w}{  }\DUrole{n}{str}}, \emph{\DUrole{n}{lon}\DUrole{p}{:}\DUrole{w}{  }\DUrole{n}{str}}, \emph{\DUrole{n}{startdato}\DUrole{p}{:}\DUrole{w}{  }\DUrole{n}{str}}, \emph{\DUrole{n}{sluttdato}\DUrole{p}{:}\DUrole{w}{  }\DUrole{n}{str}}, \emph{\DUrole{n}{para}\DUrole{p}{:}\DUrole{w}{  }\DUrole{n}{str}}}{{ $\rightarrow$ list}}
\pysigstopsignatures
\sphinxAtStartPar
Henter data frå NVE api GridTimeSeries
\begin{quote}\begin{description}
\sphinxlineitem{Parametere}\begin{itemize}
\item {} 
\sphinxAtStartPar
\sphinxstyleliteralstrong{\sphinxupquote{lat}} \textendash{} øst koordinat (i UTM33)

\item {} 
\sphinxAtStartPar
\sphinxstyleliteralstrong{\sphinxupquote{lon}} \textendash{} nord koordinat (i UTM33)

\item {} 
\sphinxAtStartPar
\sphinxstyleliteralstrong{\sphinxupquote{startdato}} \textendash{} startdato for dataserien som hentes ned

\item {} 
\sphinxAtStartPar
\sphinxstyleliteralstrong{\sphinxupquote{sluttdato}} \textendash{} sluttdato for dataserien som hentes ned

\item {} 
\sphinxAtStartPar
\sphinxstyleliteralstrong{\sphinxupquote{para}} \textendash{} kva parameter som skal hentes ned f.eks rr for nedbør

\end{itemize}

\sphinxlineitem{Returnere}
\sphinxAtStartPar
returnerer ei liste med klimaverdier

\sphinxlineitem{Retur type}
\sphinxAtStartPar
verdier

\end{description}\end{quote}

\end{fulllineitems}

\index{plot\_ekstremverdier\_3dsno() (i modul klimadata.klimadata)@\spxentry{plot\_ekstremverdier\_3dsno()}\spxextra{i modul klimadata.klimadata}}

\begin{fulllineitems}
\phantomsection\label{\detokenize{index:klimadata.klimadata.plot_ekstremverdier_3dsno}}
\pysigstartsignatures
\pysiglinewithargsret{\sphinxcode{\sphinxupquote{klimadata.klimadata.}}\sphinxbfcode{\sphinxupquote{plot\_ekstremverdier\_3dsno}}}{\emph{\DUrole{n}{df}\DUrole{p}{:}\DUrole{w}{  }\DUrole{n}{DataFrame}}, \emph{\DUrole{n}{ax1}\DUrole{o}{=}\DUrole{default_value}{None}}}{}
\pysigstopsignatures
\sphinxAtStartPar
Gammel funksjon for å jobbe med eldre tilpassing av ekstremverdiutrekning
\begin{quote}\begin{description}
\sphinxlineitem{Parametere}\begin{itemize}
\item {} 
\sphinxAtStartPar
\sphinxstyleliteralstrong{\sphinxupquote{df}} \textendash{} Pandas dataframe med klimadata

\item {} 
\sphinxAtStartPar
\sphinxstyleliteralstrong{\sphinxupquote{ax1}} \textendash{} Matplotlib axis

\end{itemize}

\sphinxlineitem{Returnere}
\sphinxAtStartPar
Ekstremverdiutrekning basert på Gumbel distribusjon

\sphinxlineitem{Retur type}
\sphinxAtStartPar
model

\end{description}\end{quote}

\end{fulllineitems}

\index{rullande\_3dogn\_nedbor() (i modul klimadata.klimadata)@\spxentry{rullande\_3dogn\_nedbor()}\spxextra{i modul klimadata.klimadata}}

\begin{fulllineitems}
\phantomsection\label{\detokenize{index:klimadata.klimadata.rullande_3dogn_nedbor}}
\pysigstartsignatures
\pysiglinewithargsret{\sphinxcode{\sphinxupquote{klimadata.klimadata.}}\sphinxbfcode{\sphinxupquote{rullande\_3dogn\_nedbor}}}{\emph{\DUrole{n}{dataframe}\DUrole{p}{:}\DUrole{w}{  }\DUrole{n}{DataFrame}}}{{ $\rightarrow$ DataFrame}}
\pysigstopsignatures
\sphinxAtStartPar
Tar in klimadataframe og returnerer med ny kollonne med utrekna 3 døgs nedbør basert på døgnnedbør
\begin{quote}\begin{description}
\sphinxlineitem{Parametere}
\sphinxAtStartPar
\sphinxstyleliteralstrong{\sphinxupquote{dataframe}} \textendash{} Pandas dataframe med klimadata

\sphinxlineitem{Returnere}
\sphinxAtStartPar
Pandas dataframe med ny kollonne med utrekna 3 døgs nedbør basert på døgnnedbør

\sphinxlineitem{Retur type}
\sphinxAtStartPar
df

\end{description}\end{quote}

\end{fulllineitems}

\index{stedsnavn() (i modul klimadata.klimadata)@\spxentry{stedsnavn()}\spxextra{i modul klimadata.klimadata}}

\begin{fulllineitems}
\phantomsection\label{\detokenize{index:klimadata.klimadata.stedsnavn}}
\pysigstartsignatures
\pysiglinewithargsret{\sphinxcode{\sphinxupquote{klimadata.klimadata.}}\sphinxbfcode{\sphinxupquote{stedsnavn}}}{\emph{\DUrole{n}{utm\_nord}\DUrole{p}{:}\DUrole{w}{  }\DUrole{n}{str}}, \emph{\DUrole{n}{utm\_ost}\DUrole{p}{:}\DUrole{w}{  }\DUrole{n}{str}}}{{ $\rightarrow$ list}}
\pysigstopsignatures
\sphinxAtStartPar
Henter stedsnavn fra geonorge api for stedsnavnsøk

\sphinxAtStartPar
Koordinatsystem er hardcoda inn i request streng sammen med søkeradius
Radius er satt til 500m
\begin{quote}\begin{description}
\sphinxlineitem{Parametere}\begin{itemize}
\item {} 
\sphinxAtStartPar
\sphinxstyleliteralstrong{\sphinxupquote{utm\_nord}} \textendash{} nord koordinat i UTM33

\item {} 
\sphinxAtStartPar
\sphinxstyleliteralstrong{\sphinxupquote{utm\_ost}} \textendash{} øst koordinat i UTM33

\end{itemize}

\sphinxlineitem{Returnere}
\sphinxAtStartPar
Liste med stedsnavn innanfor radius på 500m

\sphinxlineitem{Retur type}
\sphinxAtStartPar
verier

\end{description}\end{quote}

\end{fulllineitems}

\index{vind\_nedbor() (i modul klimadata.klimadata)@\spxentry{vind\_nedbor()}\spxextra{i modul klimadata.klimadata}}

\begin{fulllineitems}
\phantomsection\label{\detokenize{index:klimadata.klimadata.vind_nedbor}}
\pysigstartsignatures
\pysiglinewithargsret{\sphinxcode{\sphinxupquote{klimadata.klimadata.}}\sphinxbfcode{\sphinxupquote{vind\_nedbor}}}{\emph{\DUrole{n}{df}\DUrole{p}{:}\DUrole{w}{  }\DUrole{n}{DataFrame}}}{{ $\rightarrow$ DataFrame}}
\pysigstopsignatures
\sphinxAtStartPar
Tar in vinddataframe, og returnerer ny dataframe med der nedbør (rr) under 0.2mm blir fjerna
\begin{quote}\begin{description}
\sphinxlineitem{Parametere}
\sphinxAtStartPar
\sphinxstyleliteralstrong{\sphinxupquote{df}} \textendash{} Pandas dataframe med klimadata

\sphinxlineitem{Returnere}
\sphinxAtStartPar
Pandas dataframe der dager med nedbør (rr) under 0.2mm blir fjerna

\sphinxlineitem{Retur type}
\sphinxAtStartPar
df

\end{description}\end{quote}

\end{fulllineitems}

\index{vind\_regn() (i modul klimadata.klimadata)@\spxentry{vind\_regn()}\spxextra{i modul klimadata.klimadata}}

\begin{fulllineitems}
\phantomsection\label{\detokenize{index:klimadata.klimadata.vind_regn}}
\pysigstartsignatures
\pysiglinewithargsret{\sphinxcode{\sphinxupquote{klimadata.klimadata.}}\sphinxbfcode{\sphinxupquote{vind\_regn}}}{\emph{\DUrole{n}{df}\DUrole{p}{:}\DUrole{w}{  }\DUrole{n}{DataFrame}}}{{ $\rightarrow$ DataFrame}}
\pysigstopsignatures
\sphinxAtStartPar
Tar in vinddataframe, og returnerer ny dataframe med der regn (rrl) under 0.2mm blir fjerna

\sphinxAtStartPar
Bruker rrl istedenfor rr, da rr er nedbør, mens rrl er regn fra NVE Grid Times Series API
\begin{quote}\begin{description}
\sphinxlineitem{Parametere}
\sphinxAtStartPar
\sphinxstyleliteralstrong{\sphinxupquote{df}} \textendash{} Pandas dataframe med klimadata

\sphinxlineitem{Returnere}
\sphinxAtStartPar
Pandas dataframe der dager med regn (rrl) under 0.2mm blir fjerna

\sphinxlineitem{Retur type}
\sphinxAtStartPar
df

\end{description}\end{quote}

\end{fulllineitems}

\index{vind\_sno\_fsw() (i modul klimadata.klimadata)@\spxentry{vind\_sno\_fsw()}\spxextra{i modul klimadata.klimadata}}

\begin{fulllineitems}
\phantomsection\label{\detokenize{index:klimadata.klimadata.vind_sno_fsw}}
\pysigstartsignatures
\pysiglinewithargsret{\sphinxcode{\sphinxupquote{klimadata.klimadata.}}\sphinxbfcode{\sphinxupquote{vind\_sno\_fsw}}}{\emph{\DUrole{n}{df}\DUrole{p}{:}\DUrole{w}{  }\DUrole{n}{DataFrame}}}{{ $\rightarrow$ DataFrame}}
\pysigstopsignatures
\sphinxAtStartPar
Tar in vinddataframe, og returnerer ny dataframe med der nysnø under 0.2mm blir fjerna
\begin{quote}\begin{description}
\sphinxlineitem{Parametere}
\sphinxAtStartPar
\sphinxstyleliteralstrong{\sphinxupquote{df}} \textendash{} Pandas dataframe med klimadata

\sphinxlineitem{Returnere}
\sphinxAtStartPar
Pandas dataframe der dager med nysnø under 0.2mm blir fjerna

\sphinxlineitem{Retur type}
\sphinxAtStartPar
df

\end{description}\end{quote}

\end{fulllineitems}

\index{vind\_sno\_rr\_tm() (i modul klimadata.klimadata)@\spxentry{vind\_sno\_rr\_tm()}\spxextra{i modul klimadata.klimadata}}

\begin{fulllineitems}
\phantomsection\label{\detokenize{index:klimadata.klimadata.vind_sno_rr_tm}}
\pysigstartsignatures
\pysiglinewithargsret{\sphinxcode{\sphinxupquote{klimadata.klimadata.}}\sphinxbfcode{\sphinxupquote{vind\_sno\_rr\_tm}}}{\emph{\DUrole{n}{df}\DUrole{p}{:}\DUrole{w}{  }\DUrole{n}{DataFrame}}}{{ $\rightarrow$ DataFrame}}
\pysigstopsignatures
\sphinxAtStartPar
Tar in vinddataframe, og returnerer ny dataframe med der nedbør under 0.2mm og temperatur under 1 grad blir fjerna
\begin{quote}\begin{description}
\sphinxlineitem{Parametere}
\sphinxAtStartPar
\sphinxstyleliteralstrong{\sphinxupquote{df}} \textendash{} Pandas dataframe med klimadata

\sphinxlineitem{Returnere}
\sphinxAtStartPar
Pandas dataframe der dager med nedbør under 0.2mm og temperatur blir fjerna

\sphinxlineitem{Retur type}
\sphinxAtStartPar
df

\end{description}\end{quote}

\end{fulllineitems}



\chapter{Modul for plotting av klimadata}
\label{\detokenize{index:module-klimadata.plot}}\label{\detokenize{index:modul-for-plotting-av-klimadata}}\index{modul@\spxentry{modul}!klimadata.plot@\spxentry{klimadata.plot}}\index{klimadata.plot@\spxentry{klimadata.plot}!modul@\spxentry{modul}}\index{klima\_sno\_oversikt() (i modul klimadata.plot)@\spxentry{klima\_sno\_oversikt()}\spxextra{i modul klimadata.plot}}

\begin{fulllineitems}
\phantomsection\label{\detokenize{index:klimadata.plot.klima_sno_oversikt}}
\pysigstartsignatures
\pysiglinewithargsret{\sphinxcode{\sphinxupquote{klimadata.plot.}}\sphinxbfcode{\sphinxupquote{klima\_sno\_oversikt}}}{\emph{\DUrole{n}{df}}, \emph{\DUrole{n}{lokalitet}}, \emph{\DUrole{n}{annotert}}}{}
\pysigstopsignatures
\sphinxAtStartPar
Funksjonen lager sampleplot
\begin{quote}\begin{description}
\sphinxlineitem{Parametere}\begin{itemize}
\item {} 
\sphinxAtStartPar
\sphinxstyleliteralstrong{\sphinxupquote{df}} \textendash{} Dataframe med klimadata

\item {} 
\sphinxAtStartPar
\sphinxstyleliteralstrong{\sphinxupquote{lokalitet}} \textendash{} Navn på klimapunktet

\item {} 
\sphinxAtStartPar
\sphinxstyleliteralstrong{\sphinxupquote{annotert}} \textendash{} Boolsk verdi som avgjør om det skal lages annotert plot eller ikkje

\end{itemize}

\end{description}\end{quote}
\begin{description}
\sphinxlineitem{Returns}\begin{quote}
\begin{description}
\sphinxlineitem{fig}
\sphinxAtStartPar
Plott\sphinxhyphen{}objekt med 6 subplot

\end{description}
\end{quote}

\end{description}

\end{fulllineitems}

\index{klimaoversikt() (i modul klimadata.plot)@\spxentry{klimaoversikt()}\spxextra{i modul klimadata.plot}}

\begin{fulllineitems}
\phantomsection\label{\detokenize{index:klimadata.plot.klimaoversikt}}
\pysigstartsignatures
\pysiglinewithargsret{\sphinxcode{\sphinxupquote{klimadata.plot.}}\sphinxbfcode{\sphinxupquote{klimaoversikt}}}{\emph{\DUrole{n}{df}\DUrole{p}{:}\DUrole{w}{  }\DUrole{n}{DataFrame}}, \emph{\DUrole{n}{lokalitet}\DUrole{p}{:}\DUrole{w}{  }\DUrole{n}{str}}, \emph{\DUrole{n}{annotert}\DUrole{p}{:}\DUrole{w}{  }\DUrole{n}{bool}}}{{ $\rightarrow$ Figure}}
\pysigstopsignatures
\sphinxAtStartPar
Funksjonen lager sampleplot
\begin{quote}\begin{description}
\sphinxlineitem{Parametere}\begin{itemize}
\item {} 
\sphinxAtStartPar
\sphinxstyleliteralstrong{\sphinxupquote{df}} \textendash{} Dataframe med klimadata

\item {} 
\sphinxAtStartPar
\sphinxstyleliteralstrong{\sphinxupquote{lokalitet}} \textendash{} Navn på klimapunktet

\item {} 
\sphinxAtStartPar
\sphinxstyleliteralstrong{\sphinxupquote{annotert}} \textendash{} Boolsk verdi som avgjør om det skal lages annotert plot eller ikkje

\item {} 
\sphinxAtStartPar
\sphinxstyleliteralstrong{\sphinxupquote{Returns}} \textendash{} 

\item {} 
\sphinxAtStartPar
\sphinxstyleliteralstrong{\sphinxupquote{\sphinxhyphen{}\sphinxhyphen{}\sphinxhyphen{}\sphinxhyphen{}\sphinxhyphen{}\sphinxhyphen{}\sphinxhyphen{}}} \textendash{} \begin{description}
\sphinxlineitem{fig}
\sphinxAtStartPar
Plott\sphinxhyphen{}objekt med 4 subplot

\end{description}


\end{itemize}

\end{description}\end{quote}

\end{fulllineitems}

\index{normaler\_annotert() (i modul klimadata.plot)@\spxentry{normaler\_annotert()}\spxextra{i modul klimadata.plot}}

\begin{fulllineitems}
\phantomsection\label{\detokenize{index:klimadata.plot.normaler_annotert}}
\pysigstartsignatures
\pysiglinewithargsret{\sphinxcode{\sphinxupquote{klimadata.plot.}}\sphinxbfcode{\sphinxupquote{normaler\_annotert}}}{\emph{\DUrole{n}{klima}\DUrole{p}{:}\DUrole{w}{  }\DUrole{n}{DataFrame}}, \emph{\DUrole{n}{ax1}\DUrole{o}{=}\DUrole{default_value}{None}}}{{ $\rightarrow$ Axes}}
\pysigstopsignatures
\sphinxAtStartPar
Tar imot klimadataframe, og returnerer ax plotteobjekt fra matplotlib.
Funksjonen plotter normalverdier for nedbør og temperatur for perioden 1991\sphinxhyphen{}2020.
Kan kombineres i samleplot, eller stå aleine.
Finnes ein søsterfunksjon uten tall på plottet.
\begin{quote}\begin{description}
\sphinxlineitem{Parametere}
\sphinxAtStartPar
\sphinxstyleliteralstrong{\sphinxupquote{klima}} \textendash{} Klimadataframe fra klimadata.py

\sphinxlineitem{Returnere}
\sphinxAtStartPar
Plotteobjekt fra matplotlib

\sphinxlineitem{Retur type}
\sphinxAtStartPar
ax1

\end{description}\end{quote}

\sphinxAtStartPar
TODO: Kan samkøyrast med generell normalplot

\end{fulllineitems}

\index{nysnodjupne\_3d() (i modul klimadata.plot)@\spxentry{nysnodjupne\_3d()}\spxextra{i modul klimadata.plot}}

\begin{fulllineitems}
\phantomsection\label{\detokenize{index:klimadata.plot.nysnodjupne_3d}}
\pysigstartsignatures
\pysiglinewithargsret{\sphinxcode{\sphinxupquote{klimadata.plot.}}\sphinxbfcode{\sphinxupquote{nysnodjupne\_3d}}}{\emph{\DUrole{n}{df}\DUrole{p}{:}\DUrole{w}{  }\DUrole{n}{DataFrame}}, \emph{\DUrole{n}{ax1}\DUrole{o}{=}\DUrole{default_value}{None}}}{{ $\rightarrow$ Axes}}
\pysigstopsignatures
\sphinxAtStartPar
Tar inn klimadataframe og returnerer plot for 3 døgns nysnødjupne

\sphinxAtStartPar
Plottet bruker parameteren sdfsw3d \sphinxhyphen{} Nynsødybde 3 døgn frå NVE api

\sphinxAtStartPar
Se \sphinxurl{https://senorge.no/Models} for mer informasjon om måten data er generert
Det anbefales å sette seg inn i måten datasettet regner om frå mm vann til cm snø.

\sphinxAtStartPar
Plottet tar ut snittverdier for normalperiode 1961\sphinxhyphen{}1990 og 1991\sphinxhyphen{}2020 samt for heile perioden.
Det blir også rekna ut ein trend for heile datasettet.
\begin{quote}\begin{description}
\sphinxlineitem{Parametere}
\sphinxAtStartPar
\sphinxstyleliteralstrong{\sphinxupquote{df}} \textendash{} pandas dataframe med klimadata

\sphinxlineitem{Returnere}
\sphinxAtStartPar
Matplotlib axes objekt med snødjupne plot

\sphinxlineitem{Retur type}
\sphinxAtStartPar
ax1

\end{description}\end{quote}

\end{fulllineitems}

\index{plot\_aarsnedbor() (i modul klimadata.plot)@\spxentry{plot\_aarsnedbor()}\spxextra{i modul klimadata.plot}}

\begin{fulllineitems}
\phantomsection\label{\detokenize{index:klimadata.plot.plot_aarsnedbor}}
\pysigstartsignatures
\pysiglinewithargsret{\sphinxcode{\sphinxupquote{klimadata.plot.}}\sphinxbfcode{\sphinxupquote{plot\_aarsnedbor}}}{\emph{\DUrole{n}{df}\DUrole{p}{:}\DUrole{w}{  }\DUrole{n}{DataFrame}}, \emph{\DUrole{n}{ax1}\DUrole{o}{=}\DUrole{default_value}{None}}}{{ $\rightarrow$ Axes}}
\pysigstopsignatures
\sphinxAtStartPar
Tar inn klimadataframe og returnerer plot for snødjupne
Plottet bruker parameteren rr \sphinxhyphen{} Døgnnedbør v2.0 \sphinxhyphen{} mm frå NVE api
Denne returnerer døgnnedbør i mm
Se \sphinxurl{https://senorge.no/Models} for mer informasjon om måten data er generert

\sphinxAtStartPar
Plottet tar ut snittverdier for normalperiode 1961\sphinxhyphen{}1990 og 1991\sphinxhyphen{}2020 samt for heile perioden.
Det blir også rekna ut ein trend for heile datasettet.
\begin{quote}\begin{description}
\sphinxlineitem{Parametere}
\sphinxAtStartPar
\sphinxstyleliteralstrong{\sphinxupquote{df}} \textendash{} pandas dataframe med klimadata

\sphinxlineitem{Returnere}
\sphinxAtStartPar
Matplotlib axes objekt med døgnnedbør plot

\sphinxlineitem{Retur type}
\sphinxAtStartPar
ax1

\end{description}\end{quote}

\end{fulllineitems}

\index{plot\_normaler() (i modul klimadata.plot)@\spxentry{plot\_normaler()}\spxextra{i modul klimadata.plot}}

\begin{fulllineitems}
\phantomsection\label{\detokenize{index:klimadata.plot.plot_normaler}}
\pysigstartsignatures
\pysiglinewithargsret{\sphinxcode{\sphinxupquote{klimadata.plot.}}\sphinxbfcode{\sphinxupquote{plot\_normaler}}}{\emph{\DUrole{n}{klima}\DUrole{p}{:}\DUrole{w}{  }\DUrole{n}{DataFrame}}, \emph{\DUrole{n}{ax1}\DUrole{o}{=}\DUrole{default_value}{None}}}{{ $\rightarrow$ Axes}}
\pysigstopsignatures
\sphinxAtStartPar
Tar imot klimadataframe, og returnerer ax plotteobjekt fra matplotlib.
Funksjonen plotter normalverdier for nedbør og temperatur for perioden 1991\sphinxhyphen{}2020.

\sphinxAtStartPar
Kan kombineres i samleplot, eller stå aleine.
Finnes ein søsterfunksjon som skriver ut tall på plottet.
\begin{quote}\begin{description}
\sphinxlineitem{Parametere}
\sphinxAtStartPar
\sphinxstyleliteralstrong{\sphinxupquote{klima}} \textendash{} Klimadataframe fra klimadata.py

\sphinxlineitem{Returnere}
\sphinxAtStartPar
Plotteobjekt fra matplotlib

\sphinxlineitem{Retur type}
\sphinxAtStartPar
ax1

\end{description}\end{quote}

\end{fulllineitems}

\index{snodjupne() (i modul klimadata.plot)@\spxentry{snodjupne()}\spxextra{i modul klimadata.plot}}

\begin{fulllineitems}
\phantomsection\label{\detokenize{index:klimadata.plot.snodjupne}}
\pysigstartsignatures
\pysiglinewithargsret{\sphinxcode{\sphinxupquote{klimadata.plot.}}\sphinxbfcode{\sphinxupquote{snodjupne}}}{\emph{\DUrole{n}{df}\DUrole{p}{:}\DUrole{w}{  }\DUrole{n}{DataFrame}}, \emph{\DUrole{n}{ax1}\DUrole{o}{=}\DUrole{default_value}{None}}}{{ $\rightarrow$ Axes}}
\pysigstopsignatures
\sphinxAtStartPar
Tar inn klimadataframe og returnerer plot for snødjupne

\sphinxAtStartPar
Plottet bruker parameteren sd \sphinxhyphen{} Snødybde v2.0.1 frå NVE api.
Denne returnerer snødybde i cm.

\sphinxAtStartPar
Se \sphinxurl{https://senorge.no/Models} for meir informasjon om måten data er generert

\sphinxAtStartPar
Plottet tar ut snittverdier for normalperiode 1961\sphinxhyphen{}1990 og 1991\sphinxhyphen{}2020 samt for heile perioden.
Det blir også rekna ut ein trend for heile datasettet.
\begin{quote}\begin{description}
\sphinxlineitem{Parametere}
\sphinxAtStartPar
\sphinxstyleliteralstrong{\sphinxupquote{df}} \textendash{} pandas dataframe med klimadata

\sphinxlineitem{Returnere}
\sphinxAtStartPar
Matplotlib axes objekt med snødjupne plot

\sphinxlineitem{Retur type}
\sphinxAtStartPar
ax1

\end{description}\end{quote}

\end{fulllineitems}

\index{snomengde() (i modul klimadata.plot)@\spxentry{snomengde()}\spxextra{i modul klimadata.plot}}

\begin{fulllineitems}
\phantomsection\label{\detokenize{index:klimadata.plot.snomengde}}
\pysigstartsignatures
\pysiglinewithargsret{\sphinxcode{\sphinxupquote{klimadata.plot.}}\sphinxbfcode{\sphinxupquote{snomengde}}}{\emph{\DUrole{n}{df}\DUrole{p}{:}\DUrole{w}{  }\DUrole{n}{DataFrame}}, \emph{\DUrole{n}{ax1}\DUrole{o}{=}\DUrole{default_value}{None}}}{{ $\rightarrow$ Axes}}
\pysigstopsignatures
\sphinxAtStartPar
Funksjon for å plotte snomengde i løpet av året for normalperiode 1991\sphinxhyphen{}2020

\sphinxAtStartPar
Funksjonen filtrerer ut data for normalperiode 1991\sphinxhyphen{}2020.
Deretter beregnes snitt, max og min verdier for kvar dag i året, og legges til i en ny dataframe.
\begin{quote}\begin{description}
\sphinxlineitem{Parametere}\begin{itemize}
\item {} 
\sphinxAtStartPar
\sphinxstyleliteralstrong{\sphinxupquote{df}} \textendash{} Klimadataframe

\item {} 
\sphinxAtStartPar
\sphinxstyleliteralstrong{\sphinxupquote{ax1}} \textendash{} Plott\sphinxhyphen{}objekt

\end{itemize}

\sphinxlineitem{Returnere}
\sphinxAtStartPar
\begin{itemize}
\item {} 
\sphinxAtStartPar
\sphinxstyleemphasis{ax1} \textendash{} Plott\sphinxhyphen{}objekt med snødager

\item {} 
\sphinxAtStartPar
\sphinxstyleemphasis{ax2} \textendash{} Plott\sphinxhyphen{}objekt med temperatur

\end{itemize}


\end{description}\end{quote}

\end{fulllineitems}

\index{vind() (i modul klimadata.plot)@\spxentry{vind()}\spxextra{i modul klimadata.plot}}

\begin{fulllineitems}
\phantomsection\label{\detokenize{index:klimadata.plot.vind}}
\pysigstartsignatures
\pysiglinewithargsret{\sphinxcode{\sphinxupquote{klimadata.plot.}}\sphinxbfcode{\sphinxupquote{vind}}}{\emph{\DUrole{n}{vind\_df}\DUrole{p}{:}\DUrole{w}{  }\DUrole{n}{DataFrame}}}{{ $\rightarrow$ Axes}}
\pysigstopsignatures
\sphinxAtStartPar
Funksjon for å plotte vind mot nedbør og snø
\begin{description}
\sphinxlineitem{Plottet lager 3 subplots:}\begin{enumerate}
\sphinxsetlistlabels{\arabic}{enumi}{enumii}{}{.}%
\item {} 
\sphinxAtStartPar
Vindrose for vindretning uansett nedbør eller ikkje, delt inn i vindstyrker

\item {} 
\sphinxAtStartPar
Vindrose for vindretning med regn, delt inn i mm regn

\item {} 
\sphinxAtStartPar
Vindrose for vindretning med nynsø siste døgn (fsw), delt inn i cm snø

\end{enumerate}

\end{description}

\sphinxAtStartPar
Ved tolking av vindrose må ein både sjå på \% antall dager, men også på kva mengde som kjem ved kvar vindretning
det kan f.eks være flest dager frå vest, men dagene med virkelig snøfall kan komme fra andre retninger
\begin{quote}\begin{description}
\sphinxlineitem{Parametere}
\sphinxAtStartPar
\sphinxstyleliteralstrong{\sphinxupquote{vind\_df}} \textendash{} Dataframe med vinddata fra mars 2018 til mars 2022

\sphinxlineitem{Returnere}
\sphinxAtStartPar
Plott\sphinxhyphen{}objekt med 3 subplot

\sphinxlineitem{Retur type}
\sphinxAtStartPar
fig

\end{description}\end{quote}

\end{fulllineitems}



\renewcommand{\indexname}{Python Modulindex}
\begin{sphinxtheindex}
\let\bigletter\sphinxstyleindexlettergroup
\bigletter{k}
\item\relax\sphinxstyleindexentry{klimadata.klimadata}\sphinxstyleindexpageref{index:\detokenize{module-klimadata.klimadata}}
\item\relax\sphinxstyleindexentry{klimadata.plot}\sphinxstyleindexpageref{index:\detokenize{module-klimadata.plot}}
\end{sphinxtheindex}

\renewcommand{\indexname}{Index}
\printindex
\end{document}